% First we set the document class.  This is one of the standard ones.
% There are ways to create your own, or load third-party ones.

% We pass the titlepage argument, since the article class does not
% include a separate title page by default.
\documentclass{report}
\usepackage{graphicx}

\usepackage[margin=1in]{geometry}
% Let's spread the lines out a little bit.
\linespread{1.2}


\begin{document}

% Now we define the title content

% And this command actually puts the title on the page

% This is how you start a top-level section of the document.
% You don't need to close sections; a section ends when the next
% one begins, or when the document ends, whichever is first.
\newenvironment{myindentpar}[1]%
{\begin{list}{}%
    {\setlength{\leftmargin}{#1}}%
        \item[]%
}
{\end{list}}
                  
\chapter{Architectural Design}

\section{Block Diagram}
\begin{center}
\includegraphics[height=100mm]{arch_diagram.jpg}
\end{center}

\section{Command Line Interface}
To compile an RT program the following command is run:
\begin{verbatim}
	./rtc.sh <source-file> <optional destination> <optional debug>
\end{verbatim}
Running the rtc.sh script requires that a source-file is given. If no destination is provided, the compiled .class
is placed in the present working directory. If the debug flag is raised, the intermediate .java file is 
also placed into either the present working directory or the destination depending on whether one 
is provided.\\
To run the compiled .class, the following command is executed:
\begin{verbatim}
	./rt.sh <class-file> <optional args ... >
\end{verbatim}
Running the rt.sh script links the compiled .class file against the RtLib and runs executes it. 
Command line arguments can be passed to the program by appending them to the end of the call
to the script.


\end{document}
